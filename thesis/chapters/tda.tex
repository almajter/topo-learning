\chapter{Topološka analiza podatkov}

Topološka analiza podatkov (angl. Topological Data Analysis, TDA) je sodoben pristop k analizi podatkov, s katerim poskušamo razumeti obliko visokodimenzionalnih podatkovnih množic. Preučuje, kako so podatki med seboj povezani in razporejeni v prostoru, ali obstajajo kakšne povezave, zanke, praznine ter iz koliko komponent (gruč) so sestavljeni.
V praksi so se metode TDA izkazale kot zelo učinkovite pri analizi visokodimenzionalnih, razpršenih podatkov, ki vsebujejo veliko šuma. Ena izmed izmed takih metod algoritem Mapper, ki podatkovno množico pretvori v preprost simplicialni kompleks.

\section{Simpleksi}

Simpleksi so osnovni gradniki simplicialnih kompleksov. Intuitivno so $n$-simpleksi geometrijski objekti z $(n + 1)$ oglišči, ki ležijo v $n$-dimenzionalnem prostoru in jih ne moremo vstaviti v nižje dimenzionalni prostor. Za gradnjo $n$-simpleksa v $n$-dimenzionalnem prostoru potrebujemo $(n + 1)$ afino neodvisnih točk.

\begin{definicija}
    Množica točk $\{x_0, \dots, x_n\}$ v vektorskem prostoru $V$ je \textbf{afino neodvisna}, če je $\{x_1 - x_0, \dots, x_n - x_0\}$ linearno neodvisna.
\end{definicija}

S pomočjo zgornje definicije lahko simplekse matematično definiramo na naslednji način~\cite{Munkers84}.

\begin{definicija}
Naj bo \(X = \{x_0, x_1, \dots, x_n\}\) množica afino neodvisnih točk v prostoru \(\mathbb{R}^d\), kjer velja \(n \leq d\). Konveksno ovojnico množice \(X\),
\[
\operatorname{conv}(X) = \left\{ \sum_{i=0}^n \alpha_i x_i \;\middle|\; \alpha_i \in [0,1], \sum_{i=0}^n \alpha_i = 1 \right\},
\]
imenujemo \(n\)-simpleks na množici \(X\).
\end{definicija}

\begin{figure}[H]
  \centering
  \begin{tikzpicture}[scale=1.2, every node/.style={font=\small}]
  % 0-simplex (point)
  \fill[myblue!60!black] (0,0) circle (2pt);
  \node[below=4pt] at (0,-0.2) {0 - simpleks};

  % 1-simplex (line segment)
  \fill[myblue!60!black] (2,0) circle (2pt);
  \fill[myblue!60!black] (3,0) circle (2pt);
  \draw[thick] (2,0) -- (3,0);
  \node[below=4pt] at (2.5,-0.2) {1 - simpleks};

  % 2-simplex (triangle)
  \coordinate (A) at (5.5,0);
  \coordinate (B) at (6.5,0);
  \coordinate (C) at (6,1.2);
  \fill[gray!20] (A) -- (B) -- (C) -- cycle;
  \draw[thick] (A) -- (B) -- (C) -- cycle;
  \foreach \p in {A,B,C}
    \fill[myblue!20] (\p) circle (2pt);
  \node[below=4pt] at (6,-0.2) {2 - simpleks};

  % 3-simplex (tetrahedron)
  \coordinate (D) at (9,0);
  \coordinate (E) at (10,0);
  \coordinate (F) at (9.5,1.2);
  \coordinate (G) at (9.5,0.5);
  \fill[gray!20, opacity=0.5] (D) -- (E) -- (F) -- cycle;
  \draw[thick] (D) -- (E) -- (F) -- cycle;
  \draw[thick] (D) -- (G) -- (E);
  \draw[thick] (F) -- (G);
  \draw[thick] (D) -- (F);
  \draw[thick] (E) -- (F);
  \foreach \p in {D,E,F,G}
    \fill[myblue!20] (\p) circle (2pt);
  \node[below=4pt] at (9.5,-0.2) {3 - simpleks};
\end{tikzpicture}

  \caption{Simpleksi v dimenzijah 0, 1, 2 in 3}~\label{fig:basic-simplices}
\end{figure}

%Točke \(X\) imenujemo oglišča \(\sigma\), simplekse ki jih razpenjajo podmnožice \(X\) pa imenujemo lica \(\sigma\).

\section{Simplicialni kompleksi}
Simplicialni kompleksi so kombinatorični opisi topološkega prostora.
\begin{definicija}
Simplicialni kompleks \(K\) v prostoru \(R^d\) je množica simpleksov, tako da velja:
  \begin{enumerate}
    \item Vsako lice poljubnega simpleksa v simplicialnemu kompleksu \(K\) je tudi samo simpleks v \(K\).
    \item Naj bosta \(\sigma_1\) in \(\sigma_2\) poljubna simpleksa v \(K\). \v{C}e se sekata, potem je njun presek \(\sigma_1 \cap \sigma_2\) lice simpleksov \(\sigma_1\) in \(\sigma_2\) ter je simpleks v \(K\).
  \end{enumerate}
\end{definicija}

\begin{figure}[H]
  \centering
  \begin{tikzpicture}[scale=2, vertex/.style={circle, fill=black, inner sep=1.5pt}]

% === Tetrahedron faces ===
\fill[myblue!30] (1,0) -- (0.5,0.9) -- (0.6,-0.7) -- cycle;        % right face
\fill[myblue!30] (0,0) -- (1,0) -- (0.6,-0.7) -- cycle;             % bottom face
\fill[myblue!50] (0,0) -- (0.6,-0.7) -- (0.5,0.9) -- cycle;         % left face

  % === Fill the triangle on the right ===
  \fill[myblue!30] (2.4,-0.2) -- (2.9,0.4) -- (2.2,0.5) -- cycle;

  % === Tetrahedron edges ===
  \draw[dashed, myblue] (0,0) -- (1,0);              % base front edge
  \draw[thick, myblue] (0,0) -- (0.5,0.9);          % left front edge
  \draw[thick, myblue] (1,0) -- (0.5,0.9);          % right front edge
  \draw[thick, myblue] (0,0) -- (0.6,-0.7);         % left bottom edge
  \draw[thick, myblue] (0.6,-0.7) -- (0.5,0.9);     % back-left edge
  \draw[thick, myblue] (1,0) -- (0.6,-0.7);        % back-bottom hidden edge

  % === Edge between components ===
  \draw[thick, myblue] (1,0) -- (1.8,0.1);

  % === Triangle on the right ===
  \draw[thick, myblue] (2.4,-0.2) -- (2.9,0.4) -- (2.2,0.5) -- cycle;

  % === Vertical edge above the triangle ===
  \draw[thick, myblue] (2.55,0.95) -- (2.2,0.45);

  % === Vertices ===
  \foreach \x/\y in {
    0/0, 1/0, 0.5/0.9, 0.6/-0.7,
    1.8/0.1,
    2.4/-0.2, 2.9/0.4, 2.2/0.5,
    2.55/0.95
  } {
    \node[vertex] at (\x,\y) {};
  }

\end{tikzpicture}

  \caption{Simplicialni kompleks}~\label{fig:simplicial-complex}
\end{figure}

\section{Vietoris–Ripsova filtracija}
Eden najbolj uporabljenih načinov za gradnjo simplicialnih kompleksov iz podatkov je \textbf{Vietoris–Ripsov kompleks}. Postopek je sledeč:
\begin{itemize}
  \item Imamo množico točk v prostoru in izberemo parameter $\epsilon > 0$.
  \item Povežemo vsaka dva podatkovna vektorja (točki), ki sta bližje kot $\epsilon$.
  \item Če so vse povezave med trojico točk prisotne, dodamo trikotnik.
  \item Če vse povezave med četverico točk obstajajo, dodamo tetraeder itd.
\end{itemize}

S tem postopkom zgradimo simplicialni kompleks za določen $\epsilon$. Če ta parameter povečujemo, dobimo zaporedje kompleksov:
\[
K_0 \subseteq K_1 \subseteq K_2 \subseteq \cdots \subseteq K_n,
\]
kar imenujemo \textbf{filtracija}. S filtracijo opazujemo, kako se topološka struktura spreminja skozi različne vrednosti $\epsilon$.

\section{Izrek o živcu}

Na naslednji način lahko na končnem pokritju definiramo simplicialni kompleks, ki ga imenujemo živec pokritja

\begin{definicija}
Naj bo $U = \{U_\alpha\}_{\alpha \in A}$ končno pokritje prostora $X$. Živec pokritja $U$ definiramo kot simplicialni kompleks $\mathcal{N}(U)$, katerega množica oglišč je indeksna množica $A$, in kjer je družina $\{\alpha_0, \alpha_1, \dots, \alpha_k\}$ $k$-simpleks v $\mathcal{N}(U)$ natanko tedaj, ko velja $U_{\alpha_0} \cap U_{\alpha_1} \cap \dots \cap U_{\alpha_k} \ne \emptyset$.
\end{definicija}

V splošnem ne moremo trditi, da bo živec imel enake topološke lastnosti kot prostor \( X \). Ravno zato je \textit{izrek o živcu} ključen: pod določenimi pogoji zagotavlja, da sta živec \( \mathcal{N}(\mathcal{U}) \) in prostor \( X \) homotopsko ekvivalentna, tj. jih lahko zvezno preoblikujemo drug v drugega.

\begin{izrek}[Nervni izrek, Hatcher Cor. 4G.3]
Naj bo \(X\) parakomptni Hausdorffov prostor in \(\mathcal{U} = \{U_i\}_{i\in I}\) odprto pokritje prostora \(X\), v katerem so vse končne nepravilne preseke \(U_{i_0}\cap\dots\cap U_{i_k}\) kontraktibilni (tj. cover je \emph{dobar}). Potem je živec \(\mathcal{N}(\mathcal{U})\) homotopsko ekvivalenten prostoru \(X\):
\[
X \simeq \lvert\mathcal{N}(\mathcal{U})\rvert.
\]
\end{izrek}

\begin{dokaz}[Oskrbljen skica dokaz]
1. Ker je \(X\) parakomptni Hausdorffov, obstaja subtendentna \emph{particija enotnosti} \(\{\varphi_i: X\to[0,1]\}\), kjer \(\varphi_i(x)=0\) zunaj \(U_i\), in \(\sum_i\varphi_i(x)=1\) za vsak \(x\).

2. Definiramo preslikavo
\[
f: X \;\longrightarrow\; \lvert\mathcal{N}(\mathcal{U})\rvert,  
\qquad
x \;\longmapsto\; \sum_i \varphi_i(x)\,v_i,
\]
kjer so \(v_i\) oglišča ugnezdene kompleksa, pomešana s powersko kombinacijo. Zaradi kontraktibilnih presekov ta preslikava dejansko obišče samo predele, kjer je baricentričen sodoben angažma možen.

3. Hatcher dokaže, da je \(f\) homotopska ekvivalenca. Naredi jo tako, da pokaže komutativne verižnice in nato uporabi standardna topološka dejstva o homotopski ekvivalenci na pasulji.

4. Zaključimo, da je \(X\) homotopsko ekvivalenten \(\lvert\mathcal{N}(\mathcal{U})\rvert\).

Ta izrek je tako temeljna povezava med pokritjem prostora in simplicialnim povzetkom oblike.
\end{dokaz}


\section{Vztrajna homologija}
Vztrajna homologija je metoda, ki sledi, kako se topološke značilnosti (komponente, zanke, votline) pojavljajo in izginjajo skozi filtracijo.

Za vsako dimenzijo $k$ štejemo $k$-dimenzionalne luknje:
\begin{itemize}
  \item V dimenziji 0: koliko ločenih komponent obstaja.
  \item V dimenziji 1: koliko zank ali obročev.
  \item V dimenziji 2: koliko votlih prostorov.
\end{itemize}

Ko povečujemo $\epsilon$, se nekatere značilnosti pojavijo (rojstvo), druge izginejo (smrt). Zabeležimo jih v obliki:
\[
(\epsilon_\text{rojstvo}, \epsilon_\text{smrt})
\]

Daljši kot je razmik med rojstvom in smrtjo, bolj \textbf{vztrajna} je značilnost. Takšne značilnosti veljajo za pomembne. Te podatke vizualiziramo kot:
\begin{itemize}
  \item \textbf{Vztrajni diagram} – točke nad diagonalo $(\epsilon_\text{rojstvo}, \epsilon_\text{smrt})$.
  \item \textbf{Črtne kode} – vsaka značilnost je predstavljena kot daljica.
\end{itemize}

Persistentna homologija nam omogoča izluščiti robustne strukturne značilnosti podatkov in jih ločiti od šuma. Skupaj z Vietoris–Ripsovo filtracijo tvori eno osrednjih orodij v TDA.

\section{Mapper}
Primer v praksi: Odmeven primer uspešne uporabe TDA prihaja s področja biomedicine, kjer je topološka analiza razkrila pomemben nov vzorec v genomskih podatkih raka dojk. Nicolau s sodelavci so leta 2011 uporabili Mapper (v okviru metode Progression Analysis of Disease) za vizualizacijo transkripcijskih mikroarray podatkov tumorjev dojk
pmc.ncbi.nlm.nih.gov
. Rezultat je bil graf s tremi „vejnami“, ki predstavljajo različne podtipe raka – med njimi so avtorji odkrili novo podskupino znotraj estrogensko pozitivnih tumorjev (ER+), označeno z izrazito povišano ekspresijo gena c-MYB in znižano ekspresijo skupine vnetnih genov
pmc.ncbi.nlm.nih.gov
. Presenetljivo je, da so imele pacientke v tej skupini 100 % stopnjo preživetja in praktično brez metastaz, kar kaže na izjemno nenevaren potek bolezni
pmc.ncbi.nlm.nih.gov
. Analiza je to skupino izluščila povsem iz vzorcev genske ekspresije, brez uporabe kakršnih koli vnaprejšnjih kliničnih informacij (npr. podatkov o izidu bolezni)
pmc.ncbi.nlm.nih.gov
. Pomembno je poudariti, da je bila ta podskupina pred uporabo TDA nevidna za klasično gručenje – pri običajni grozdni analizi so bili namreč ti tumorji razpršeni med več različnih gruč, zaradi česar njihov skupni signal ni izstopal
pmc.ncbi.nlm.nih.gov
. Topološka analiza pa je razkrila, da ti tumorji tvorijo ločeno, izjemno homogeno skupino z jasnim molekularnim podpisom, ki ne ustreza nobenemu od prej znanih podtipov (ni šlo ne za luminalni A/B ne za t. i. normal-like rak)
pmc.ncbi.nlm.nih.gov
. Novo odkrito skupino so po glavni značilnosti poimenovali c-MYB+ podtip raka dojk
pmc.ncbi.nlm.nih.gov
. Ta primer ponazarja, kako lahko TDA v biomedicinskih podatkih odkrije pomembne skrite vzorce, ki jih tradicionalne metode ne zaznajo, bodisi zaradi šuma bodisi zaradi kompleksnosti podatkov
pmc.ncbi.nlm.nih.gov
. Topološka analiza podatkov se tako vse bolj izkazuje kot obetaven pristop za robustno analizo kompleksnih, visokodimenzionalnih podatkovnih množic v znanosti in inženirstvu.

\begin{algorithm}
\caption{Mapper}\label{alg:mapper2}
\begin{algorithmic}
\State{} \textbf{Vhod:} Zbirka (množica) podatkov $X$ z metriko, funkcija $f: X \rightarrow \mathbb{R}$ (ali $\mathbb{R}^d$), in pokritje $\mathcal{U}$ prostora $f(X)$.
\State{} \textbf{Algoritem:}
\For{each $U \in \mathcal{U}$}
    \State{} Razdeli $f^{-1}(U)$  v skupine $C_{U,1}, \ldots, C_{U,k_U}$.
    \State{} Izračunaj živec pokritja $X$, ki ga definira $\{C_{U,1}, \ldots, C_{U,k_U}\}$ for each $U \in \mathcal{U}$.
\EndFor{}
\State{} \textbf{Izhod:} Simplicialni kompleks, živec (pogosto graf za dobro izbrana pokritja):
\State{} \quad – Vozlišče $v_{U,i}$ za vsako skupino $C_{U,i}$.
\State{} \quad – Povezava $v_{U,i}$ in $v_{U',j}$ če $C_{U,i} \cap C_{U',j} \neq \emptyset$.
\end{algorithmic}
\end{algorithm}

\begin{algorithm}
\caption{Algoritem Mapper}\label{alg:mapper}
\begin{algorithmic}[1]
\State \textbf{Vhod:} Množica podatkov $X$ z metrikom, 
    zvezna funkcija $f: X \rightarrow \mathbb{R}$ (ali $f: X \rightarrow \mathbb{R}^d$), 
    in odprto pokritje $\mathcal{U} = \{U_\alpha\}$ prostora $f(X)$.
\State \textbf{Postopek:}
\For{vsak $U_\alpha \in \mathcal{U}$}
    \State Izračunaj preddsliko $f^{-1}(U_\alpha) \subseteq X$.
    \State Na $f^{-1}(U_\alpha)$ uporabi metodo gručenja (npr. single-linkage) in razdeli v gruče $C_{\alpha,1}, \ldots, C_{\alpha,k_\alpha}$.
\EndFor
\State \textbf{Zgradi simplicialni kompleks (živec pokritja):}
\State \quad Vozlišča ustrezajo posameznim gručam $C_{\alpha,i}$.
\State \quad Poveži dve vozlišči $C_{\alpha,i}$ in $C_{\beta,j}$ z robom, če velja $C_{\alpha,i} \cap C_{\beta,j} \neq \emptyset$.
\State \textbf{Izhod:} Simplicialni kompleks (pogosto graf), ki povzame topološko strukturo podatkov.
\end{algorithmic}
\end{algorithm}