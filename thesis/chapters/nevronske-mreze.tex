\chapter{Nevronske mreže}

V tem poglavju bomo podrobneje predstavili naprej usmerjene (angl.\ feedforward) umetne nevronske mreže, natančneje večnivojske perceptrone (angl.\ multilayer perceptron). Zaradi boljše berljivosti bomo v nadaljevanju uporabljali zgolj izraz nevronska mreža. Najprej bomo predstavili osnovne koncepte, nato podali matematični model ter opisali algoritem vzvratnega razširjanja napake.

%%%%%%%%%%%%%%%%%%%%%%%%%%%%%%%%%%%%%%%%%%%%%%%%%%%%%%%%%%%%%%%%%%%%%%%%%%%%%%%%%%%%%%%%%%%%%

\section{Osnovni koncepti}

Umetna nevronska mreža je matematični model, ki posnema delovanje človeških možganov. Predstavljamo si jo lahko kot usmerjen acikličen graf, kjer so vozlišča oziroma nevroni urejeni v zaporedne sloje. Povezave med njimi so utežene in potekajo izključno med nevroni sosednjih slojev, v smeri od prvega (vhodnega) sloja proti zadnjemu (izhodnemu) sloju. Sloje običajno razdelimo na tri tipe:
\begin{itemize}
  \item \textbf{Vhodni sloj} je prvi sloj, ki sprejme vhodne podatke in jih brez sprememb posreduje skritemu sloju. Število nevronov v tem sloju ustreza dimenziji vhodne množice.
  \item \textbf{Skriti sloji} se nahajajo med vhodnim in izhodnim slojem ter so odgovorni za večino obdelave podatkov. Njihovo število in velikost se lahko poljubno prilagajata glede na kompleksnost problema.
  \item \textbf{Izhodni sloj} je zadnji sloj in vrne končni rezultat, ki predstavlja napoved mreže. Število nevronov v tem sloju je odvisno od vrste problema – pri regresiji je to običajno en nevron, pri klasifikaciji pa en nevron za vsak razred. 
\end{itemize}

\begin{figure}[H]
  \centering
  % LAYERS IN NEURAL NETWORK
\begin{tikzpicture}[scale=1.25,x=2.2cm,y=1.4cm]

  \message{^^Layers in Neural network}
  \readlist\Nnod{4,5,5,5,3} % array of number of nodes per layer
  \readlist\Nstr{n,m,m,m,k} % array of string number of nodes per layer
  % \readlist\Cstr{\strut x,a^{(\prev)},a^{(\prev)},a^{(\prev)},y} % array of coefficient symbol per layer
  \def\yshift{0.5} % shift last node for dots
  
  \message{^^Drawing Layers}
  \foreachitem \N \in \Nnod{ % loop over layers
    \def\lay{\Ncnt} % alias of index of current layer
    \pgfmathsetmacro\prev{int(\Ncnt-1)} % number of previous layer
    \message{\lay,}
    \foreach \i [evaluate={\c=int(\i==\N); \y=\N/2-\i-\c*\yshift;
                 \index=(\i<\N?int(\i):"\Nstr[\lay]");
                 \x=\lay; \n=\nstyle;}] in {1,...,\N}{ % loop over nodes
      % NODES
      \node[node \n] (N\lay-\i) at (\x,\y) {}; % {$\Cstr[\lay]_{\index}$};
      
      % CONNECTIONS
      \ifnum\lay>1 % connect to previous layer
        \foreach \j in {1,...,\Nnod[\prev]}{ % loop over nodes in previous layer
          \draw[connect arrow, gray] (N\prev-\j) -- (N\lay-\i); % connect arrows directly
        }
      \fi % else: nothing to connect first layer
      
    }
    \path (N\lay-\N) --++ (0,1+\yshift) node[midway,scale=1.5] {$\vdots$};
  }
  
  % RECTANGLE
  \node[
    draw=myblue!40,
    fill=myblue,
    fill opacity=0.02,
    rounded corners=2,
    inner sep=10pt,
    fit=(N2-1)(N4-5)
  ] {};

  % LABELS
  \node[above=5,align=center,softorange!60!black] at (N1-1.90) {vhodni\\[-0.2em]sloj};
  \node[above=10,align=center,myblue!60!black] at (N3-1.90) {skriti sloji};
  \node[above=5,align=center,myred!60!black] at (N\Nnodlen-1.90) {izhodni\\[-0.2em]sloj};
  
\end{tikzpicture}

  \caption{Sloji v nevronski mreži}~\label{fig:nn-layers}
\end{figure}

Izhodno vrednost nevrona imenujemo aktivacija. Vsak nevron sprejme aktivacije nevronov iz prejšnjega sloja, oziroma vhodne podatke v primeru prvega sloja. Iz teh vrednosti izračuna uteženo vsoto, nato pa rezultat preslika s t.\  i.\ aktivacijsko funkcijo. Aktivacijska funkcija je praviloma nelinearna, kar zagotavlja nelinearnost modela.
Da bi utežena vsota lahko vsebovala konstantni člen, vsakemu sloju razen izhodnega dodamo po en nevron s konstantno aktivacijo ena.

\begin{figure}[H]
  \centering
  \tikzset{
  connect arrow/.style={->, draw=gray}
}

% LAYERS IN NEURAL NETWORK
\begin{tikzpicture}[scale=1.25,x=2.2cm,y=1.4cm]

  \message{^^Layers in Neural network}
  \readlist\Nnod{4,5,5,5,3} % array of number of nodes per layer
  \readlist\Nstr{n,m,m,m,k} % array of string number of nodes per layer
  % \readlist\Cstr{\strut x,a^{(\prev)},a^{(\prev)},a^{(\prev)},y} % array of coefficient symbol per layer
  \def\yshift{0.5} % shift last node for dots
  
  \message{^^Drawing Layers}
  \foreachitem \N \in \Nnod { % loop over layers
    \def\lay{\Ncnt} % current layer index
    \pgfmathsetmacro\prev{int(\Ncnt - 1)} % previous layer index
    \message{\lay,}

    % Number of total nodes in this layer including bias if not last
    \ifnum\lay<\Nnodlen
      \pgfmathsetmacro\Nwithbias{\N + 1}
      \def\firstIndex{0}
    \else
      \pgfmathsetmacro\Nwithbias{\N}
      \def\firstIndex{1}
    \fi

    % All neurons in layer: bias (index 0) + regular (1..N)
    \foreach \i [evaluate={
        \c=int(\i==\N);
        \y=\N/2-\i-\c*\yshift;
        \x=\lay; \n=\nstyle;
      }] in {\firstIndex,...,\N} {

      % \node[node \n] (N\lay-\i) at (\x,\y) {};

      \ifnum\i=0
        % \node[circle, draw=teal!60!black, fill=teal!20, minimum size=8pt, inner sep=2pt] (N\lay-\i) at (\x,\y) {};
        \node[node bias] (N\lay-\i) at (\x,\y) {+1};
      \else
        \node[node \n] (N\lay-\i) at (\x,\y) {};
      \fi

      % Only connect if NOT bias node
      \ifnum\i>0
        \ifnum\lay>1
          \ifnum\prev<\Nnodlen
            \foreach \j in {0,...,\Nnod[\prev]} {
              \draw[connect arrow] (N\prev-\j) -- (N\lay-\i);
            }
          \else
            \foreach \j in {1,...,\Nnod[\prev]} {
              \draw[connect arrow] (N\prev-\j) -- (N\lay-\i);
            }
          \fi
        \fi
      \fi
    }

    % Dots
    \path (N\lay-\N) --++ (0,1+\yshift) node[midway,scale=1.5] {$\vdots$};
  }
  
  % RECTANGLE
  % \node[
  %   draw=myblue!40,
  %   fill=myblue,
  %   fill opacity=0.02,
  %   rounded corners=2,
  %   inner sep=10pt,
  %   fit=(N2-1)(N4-5)
  % ] {};

  % LABELS
  % \node[above=5,align=center,softorange!60!black] at (N1-1.90) {vhodni\\[-0.2em]sloj};
  % \node[above=10,align=center,myblue!60!black] at (N3-1.90) {skriti sloji};
  % \node[above=5,align=center,myred!60!black] at (N\Nnodlen-1.90) {izhodni\\[-0.2em]sloj};
  
\end{tikzpicture}

  \caption{Nevron s konstantno aktivacijo}~\label{fig:nn-bias}
\end{figure}

%%%%%%%%%%%%%%%%%%%%%%%%%%%%%%%%%%%%%%%%%%%%%%%%%%%%%%%%%%%%%%%%%%%%%%%%%%%%%%%%%%%%%%%%%%%%%

\section{Matematični model}

V nadaljevanju bomo definirali notacijo, ki je pogosto uporabljena v strokovni literaturi s področja umetne inteligence, na primer v~\cite{Hastie2009}. Takšna notacija omogoča dosledno in formalno predstavitev nevronskih mrež.

Naj bo nevronska mreža sestavljena iz $k+1$ slojev, označenih z $L^{(0)}, L^{(1)}, \dots, L^{(k)}$, kjer je $L^{(0)}$ vhodni sloj, $L^{(k)}$ pa izhodni sloj. Vsak sloj $L^{(i)}$ vsebuje $N^{(i)}$ nevronov, za $i = 0, \dots, k$.

Vhod v mrežo predstavimo z vektorjem $x \in \mathbb{R}^{N^{(0)}}$, ki vsebuje $N^{(0)}$ komponent in služi kot začetna aktivacija v vhodnem sloju.

Za sloje nevronske mreže uvedemo naslednje pojme:

\begin{itemize}
  \item $a^{(i)}$ aktivacijska funkcija v sloju $L^{(i)}$, za $i = 1, \dots, k$.
  
  \item $h^{(i)} = \left(h^{(i)}_1, h^{(i)}_2, \dots, h^{(i)}_{N^{(i)}}\right)$ je vektor aktivacij v sloju $L^{(i)}$, za $i = 0, \dots, k$. Za vhodni sloj velja $h^{(0)} = x$.
  
  \item $z^{(i)} = \left(z^{(i)}_1, z^{(i)}_2, \dots, z^{(i)}_{N^{(i)}}\right)$ je vektor uteženih vsot v sloju $L^{(i)}$, za $i = 1, \dots, k$, pri čemer vsaka komponenta predstavlja vsoto uteženih vhodov za posamezen nevron:
  \[
    z^{(i)}_j = \sum_{\ell=1}^{N^{(i-1)}} w^{(i)}_{j\ell} h^{(i-1)}_\ell + b^{(i)}_j, \quad \text{za } j = 1, \dots, N^{(i)}.
  \]

  \item $W^{(i)}$ je matrika uteži dimenzije $N^{(i)} \times N^{(i-1)}$, ki povezuje sloj $L^{(i-1)}$ s slojem $L^{(i)}$, za $i = 1, \dots, k$. Ima naslednjo obliko:
  \[
    W^{(i)} = \begin{bmatrix}
      w^{(i)}_{1,1} & w^{(i)}_{1,2} & \cdots & w^{(i)}_{1,N^{(i-1)}} \\
      w^{(i)}_{2,1} & w^{(i)}_{2,2} & \cdots & w^{(i)}_{2,N^{(i-1)}} \\
      \vdots       & \vdots       & \ddots & \vdots               \\
      w^{(i)}_{N^{(i)},1} & w^{(i)}_{N^{(i)},2} & \cdots & w^{(i)}_{N^{(i)},N^{(i-1)}}
    \end{bmatrix}
  \]

  \item $b^{(i)} = \left(b^{(i)}_1, b^{(i)}_2, \dots, b^{(i)}_{N^{(i)}}\right)$ je vektor uteži nevronov s konstantno aktivacijo (angl.\ bias) za sloj $L^{(i)}$, za $i = 1, \dots, k$.
\end{itemize}

Aktivacije v sloju $L^{(i)}$ izračunamo s pomočjo aktivacijske funkcije $a^{(i)}$, ki se uporablja po komponentah:
  \[
    h^{(i)}_j = a^{(i)}\left(z^{(i)}_j\right), \quad \text{za } j = 1, \dots, N^{(i)}.
  \]

Postopek se iterativno izvede za sloje $i = 1, \dots, k$, pri čemer vektor $h^{(k)}$ predstavlja izhod mreže.

TODO:

Povratno razširjanje (angl.~\textit{backpropagation}) je metoda za učenje v umetnih nevronskih mrežah, ki temelji na minimizaciji funkcije izgube z gradientnim spustom. Širšo pozornost je pritegnila v delu Rumelharta, Hintona in Williamsa leta 1986~\cite{rumelhart1986learning}.

Naj bo \( X \in \mathbb{R}^{N \times d} \) matrika vhodnih podatkov, kjer vsak vhodni primer \( x^{(i)} \in \mathbb{R}^d \) predstavlja vrstico, in \( y \in \mathbb{R}^{N \times c} \) matrika pripadajočih ciljev, kjer je \( c \) število razredov. Učni cilj je minimizirati povprečno funkcijo izgube:
\[
\mathcal{J}(W) = \frac{1}{N} \sum_{i=1}^N \mathcal{L}(h^{(k)}(x^{(i)}), y^{(i)}),
\]
kjer je \( h^{(k)}(x^{(i)}) \) izhod nevronske mreže s \( k \) plastmi za vhod \( x^{(i)} \).

Aktivacije v sloju \( L^{(i)} \) izračunamo s pomočjo aktivacijske funkcije \( a^{(i)} \), ki deluje po komponentah:
\[
h^{(i)}_j = a^{(i)}(z^{(i)}_j), \quad \text{za } j = 1, \dots, N^{(i)},
\]
kjer je \( z^{(i)} = W^{(i)} h^{(i-1)} + b^{(i)} \), začetna aktivacija pa je \( h^{(0)} = x \).

Ta postopek definiramo rekurzivno:
\[
h^{(i)} = a^{(i)}(W^{(i)} h^{(i-1)} + b^{(i)}), \quad \text{za } i = 1, \dots, k,
\]
in končni izhod mreže je \( h^{(k)} \). To rekurzivno predstavitev bomo kasneje uporabili pri formulaciji algoritma povratnega razširjanja.


Za dano zaporedje parametrov \( \{(W^{(i)}, b^{(i)}, a^{(i)})\}_{i=1}^k \), kjer je \( W^{(i)} \) matrika uteži, \( b^{(i)} \) vektor pristranskosti in \( a^{(i)} \) aktivacijska funkcija za plast \( i \), definiramo izhod mreže za vhod \( x \in \mathbb{R}^d \) z naslednjo rekurzivno relacijo:
\[
h^{(0)} = x, \quad h^{(i)} = a^{(i)}(W^{(i)} h^{(i-1)} + b^{(i)}), \quad \text{za } i = 1, \dots, k.
\]
Končni izhod mreže je tako \( h^{(k)} \in \mathbb{R}^{N^{(k)}} \), kjer \( N^{(k)} \) določa število izhodnih nevronov.

To rekurzivno strukturo bomo kasneje izkoristili pri izpeljavi povratnega razširjanja, kjer bomo izraze \( \frac{\partial \mathcal{L}}{\partial W^{(i)}} \) računali v obratnem vrstnem redu.


\begin{figure}[H]
  \centering
  \scalebox{0.85}{% NEURAL NETWORK activation with bias
\begin{tikzpicture}[x=2.7cm,y=1.6cm]

  \message{^^JNeural network activation}
  \def\NI{5} % number of regular nodes in input layer
  \def\NO{4} % number of nodes in output layer
  \def\yshift{0.4} % shift last node for dots

  % Total number of input nodes including bias
  \pgfmathsetmacro{\NItotal}{\NI + 1}

  % INPUT LAYER: Bias is node 0, rest from 1 to \NI
  \foreach \i [evaluate={
      \c=int(\i==\NI);
      \y = \NItotal/2 - \i - \c*\yshift;
      \index = (\i==0 ? "b" : (\i==\NI ? "N^0" : int(\i)));
    }] in {0,...,\NI} {
    
    \ifnum\i=0
      % Bias node
      \node[node bias] (NI-\i) at (0,\y) {$1$};
    \else
      % Regular input nodes
      \node[node in,outer sep=0.6] (NI-\i) at (0,\y) {$h_{\index}^{(0)}$};
    \fi
  }

  % OUTPUT LAYER
  \foreach \i [evaluate={\c=int(\i==\NO); \y=\NO/2-\i-\c*\yshift; \index=(\i<\NO?int(\i):"N^1");}]
           in {\NO,...,1} {

    % Highlighted first output neuron
    \ifnum\i=1
      \node[node hidden] (NO-\i) at (1,\y) {$h_{\index}^{(1)}$};
      \foreach \j [evaluate={\index=(\j<\NI?int(\j):"N^0");}] in {0,...,\NI} {
      \draw[connect arrow,white,line width=1.2] (NI-\j) -- (NO-\i);
      \ifnum\j=0
        \draw[connect arrow] (NI-\j) -- (NO-\i)
          node[pos=0.5] {\contour{white}{$b_{1}^{(1)}$}};
      \else
        \draw[connect arrow] (NI-\j) -- (NO-\i)
          node[pos=0.50] {\contour{white}{$w^{(1)}_{1,\index}$}};
      \fi
    }

    \else
      \node[node,blue!20!black!80,draw=myblue!20,fill=myblue!5]
        (NO-\i) at (1,\y) {$h_{\index}^{(1)}$};
      \foreach \j in {0,...,\NI} {
        \draw[connect arrow,gray] (NI-\j) -- (NO-\i);
      }
    \fi
  }

  % DOTS
  \path (NI-\NI) --++ (0,1+\yshift) node[midway,scale=1.2] {$\vdots$};
  \path (NO-\NO) --++ (0,1+\yshift) node[midway,scale=1.2] {$\vdots$};

  % EQUATIONS
  \def\agr#1{{\color{myorange}h_{#1}^{(0)}}}
  \node[below=16,right=11,mydarkblue,scale=0.95] at (NO-1)
    {$\begin{aligned}
       &= \color{mydarkred}a^{(1)}\left( \color{black}
            w^{(1)}_{1,1}\agr{1} + w^{(1)}_{1,2}\agr{2} + \ldots + w^{(1)}_{1,N^0}\agr{n} + b_1^{(1)}
          \color{mydarkred}\right)\\
       &= \color{mydarkred}a^{(1)}\left( \color{black}
            \sum_{i=1}^{n} w^{(1)}_{1,i}\agr{i} + b_1^{(1)}
           \color{mydarkred}\right)= a^{(1)}(z^{(1)})
     \end{aligned}$};
  \node[right,scale=0.9] at (1.3,-1.3)
    {$\begin{aligned}
      {\color{mydarkblue}
      \begin{pmatrix}
        h_{1}^{(1)} \\[0.3em]
        h_{2}^{(1)} \\
        \vdots \\
        h_{N^1}^{(1)}
      \end{pmatrix}}
      &=
      \color{mydarkred}a^{(1)}\left[ \color{black}
      \begin{pmatrix}
        w^{(1)}_{1,1} & w^{(1)}_{1,2} & \ldots & w^{(1)}_{1,N^0} \\
        w^{(1)}_{2,1} & w^{(1)}_{2,2} & \ldots & w^{(1)}_{2,N^0} \\
        \vdots  & \vdots  & \ddots & \vdots  \\
        w^{(1)}_{N^1,1} & w^{(1)}_{N^1,2} & \ldots & w^{(1)}_{N^1,N^0}
      \end{pmatrix}
      {\color{myorange}
      \begin{pmatrix}
        h_{1}^{(0)} \\[0.3em]
        h_{2}^{(0)} \\
        \vdots \\
        h_{N^0}^{(0)}
      \end{pmatrix}}
      +
      \begin{pmatrix}
        b_{1}^{(1)} \\[0.3em]
        b_{2}^{(1)} \\
        \vdots \\
        b_{N^1}^{(1)}
      \end{pmatrix}
      \color{mydarkred}\right]\\[0.5em]
      {\color{mydarkblue}\mathbf{h}^{(1)}}
      &= \color{mydarkred}a^{(1)}\left( \color{black}
           \mathbf{W}^{(1)} {\color{myorange}\mathbf{h}^{(0)}}+\mathbf{b}^{(1)}
         \color{mydarkred}\right)
    \end{aligned}$};

\end{tikzpicture}
} % scale to 85%
  \caption{Aktivacija nevrona v prvem sloju}~\label{fig:nn-activation}
\end{figure}

%%%%%%%%%%%%%%%%%%%%%%%%%%%%%%%%%%%%%%%%%%%%%%%%%%%%%%%%%%%%%%%%%%%%%%%%%%%%%%%%%%%%%%%%%%%%%

\subsection*{Aktivacijske funkcije}

Aktivacijske funkcije so nelinearne funkcije, ki preslikajo uteženo vsoto nevronov aktivacij iz prejšnjega sloja.

Aktivacijske funkcije določajo, kako se izhod posameznega nevrona iz pretekle plasti pretvori v vhod za naslednjo plast. Brez nelinearnih aktivacij bi bile nevronske mreže le linearni modeli, nezmožni učinkovitega učenja kompleksnih vzorcev.

\subsection*{Pogoste aktivacijske funkcije}

\begin{itemize}
  \item Sigmoidna funkcija:
  \[
  \sigma(x) = \frac{1}{1 + e^{-x}}
  \]
  Pretvori vhod v vrednosti med 0 in 1. Pogosto se uporablja v izhodni plasti za binarno klasifikacijo.
  \begin{figure}[H]
\centering
\begin{tikzpicture}
\begin{axis}[domain=-10:10, samples=200, axis lines=middle, xlabel=$x$, ylabel={$y$}, title={Sigmoid}]
\addplot[blue, thick] {1/(1 + exp(-x))};
\end{axis}
\end{tikzpicture}
\end{figure}

  \item Tanh funkcija:
  \[
  \tanh(x) = \frac{e^x - e^{-x}}{e^x + e^{-x}}
  \]
  Vrednosti gredo od -1 do 1. Ima boljšo povprečno izhodno vrednost kot sigmoid.
\begin{figure}[h]
\centering
\begin{tikzpicture}
\begin{axis}[domain=-10:10, samples=200, axis lines=middle, xlabel=$x$, ylabel={$y$}, title={Tanh}]
\addplot[red, thick] {tanh(x)};
\end{axis}
\end{tikzpicture}
\end{figure}

  \item ReLU (Rectified Linear Unit):
  \[
  \text{ReLU}(x) = \max(0, x)
  \]
  Najpogosteje uporabljena zaradi učinkovitosti in reševanja problema izginjajočih gradientov.
\begin{figure}[H]
\centering
\begin{tikzpicture}
\begin{axis}[domain=-10:10, samples=200, axis lines=middle, xlabel=$x$, ylabel={$y$}, title={ReLU}]
\addplot[green, thick] {max(0,x)};
\end{axis}
\end{tikzpicture}
\end{figure}

  \item Leaky ReLU:
  \[
  \text{Leaky ReLU}(x) = \max(\alpha x, x), \quad \alpha > 0
  \]
  Omogoča majhen negativni gradient tudi, ko je vhod negativen.
  \begin{figure}[H]
\centering
\begin{tikzpicture}
\begin{axis}[domain=-10:10, samples=200, axis lines=middle, xlabel=$x$, ylabel={$y$}, title={Leaky ReLU}]
\addplot[orange, thick] {x < 0 ? 0.1*x : x};
\end{axis}
\end{tikzpicture}
\end{figure}


  \item Softmax:
  \[
  \text{softmax}(x_i) = \frac{e^{x_i}}{\sum_j e^{x_j}}
  \]
  Uporablja se v izhodni plasti za večrazredno klasifikacijo, saj pretvori vrednosti v verjetnosti.
\end{itemize}
Izbor aktivacijske funkcije pomembno vpliva na učno dinamiko. Nelinearnost omogoča, da mreža modelira kompleksne relacije. Derivativ aktivacijske funkcije se neposredno uporablja pri povratnem razširjanju za izračun gradientov.

%%%%%%%%%%%%%%%%%%%%%%%%%%%%%%%%%%%%%%%%%%%%%%%%%%%%%%%%%%%%%%%%%%%%%%%%%%%%%%%%%%%%%%%%%%%%%

\section{Vzvratno razširjanje napake}

Vzvratno razširjanje (angl.~\textit{backpropagation}) napake je metoda za učenje nevronskih mrež, ki temelji na minimizaciji funkcije napake z gradientnim spustom. Spada med metode nadozorovanega učenja, kjer imamo podano učno množico podatkov s pripadajočimi ciljnimi vrednostmi. Širšo pozornost je pritegnila z delom Rumelharta, Hintona in Williamsa leta 1986~\cite{Rumelhart1986}.

Naj bo podana učna množica s \( p \in \mathbb{N} \) primeri kot matrika \( X \in \mathbb{R}^{p \times N^{(0)}} \), kjer vsak vrstični vektor \( x_i \in \mathbb{R}^{N^{(0)}} \), za \( i = 1, \dots, p \), predstavlja en vhodni primer, in \( y \in \mathbb{R}^{p \times N^{(k)}} \) matrika pripadajočih ciljnih vrednosti.

Uteži mreže na začetku inicializiramo naključno, nato pa jih z učenjem postopoma prilagajamo.

Cilj učenja je najti tak nabor uteži, da bo izhodni vektor nevronske mreže za vsak vhod čim bolj podoben želenemu izhodu. To dosežemo z minimizacijo kriterijske funkcije oz.\ funkcije napake \( E \), ki jo podamo kot:
\[
E = \frac{1}{2} \sum_c \sum_j (y_{j,c} - d_{j,c})^2,
\]

kjer je \( y_{j,c} \) izhod mreže in \( d_{j,c} \) ustrezna ciljna vrednost.

% THIS IS NEW CHAPTER

Cilj učenja je najti tak nabor uteži, da bo izhodni vektor nevronske mreže za vsak vhod čim bolj podoben želenemu izhodu. Če imamo končno množico vhodno-izhodnih parov, lahko skupno napako ocenimo s primerjavo dejanskih in ciljnih vrednosti. Skupna napaka \( E \) je definirana kot:
\[
E = \frac{1}{2} \sum_c \sum_j (y_{j,c} - d_{j,c})^2,
\]
kjer je \( c \) indeks primera, \( j \) indeks izhodne enote, \( y_{j,c} \) dejanski izhod in \( d_{j,c} \) želeni izhod.

Za učenje z gradientnim spustom moramo izračunati parcialne odvode napake glede na vse uteži v mreži. To izvedemo v dveh fazah:
\begin{enumerate}
  \item \textbf{Propagacija naprej}: izračun aktivacij v vsaki plasti na osnovi vhodov.
  \item \textbf{Propagacija nazaj}: uporaba verižnega pravila za izračun gradientov.
\end{enumerate}

\begin{enumerate}
  \item \textbf{Propagacija naprej:} za dan vhodni vektor $x$ izračunamo aktivacije vseh slojev $h^{(1)}, h^{(2)}, \dots, h^{(k)}$ in dobimo izhod modela $h^{(k)}$.
  \item \textbf{Propagacija nazaj:} izhod primerjamo s ciljno vrednostjo $y$ in izračunamo napako. Nato postopno računamo gradient funkcije napake $E$ po slojih navzdol, pri čemer za vsak sloj $i$ izračunamo odvod $\frac{\partial E}{\partial W^{(i)}}$ in $\frac{\partial E}{\partial b^{(i)}}$.
\end{enumerate}

Za izhodne enote imamo:
\[
\frac{\partial E}{\partial y_j} = y_j - d_j,
\]
\[
\frac{\partial E}{\partial x_j} = (y_j - d_j) \cdot y_j (1 - y_j).
\]

Za uteži \( w_{ji} \) med enotama \( i \) in \( j \) dobimo:
\[
\frac{\partial E}{\partial w_{ji}} = \frac{\partial E}{\partial x_j} \cdot y_i.
\]

Za enoto \( j \) iz predzadnje plasti velja:
\[
\frac{\partial E}{\partial y_j} = \sum_k \frac{\partial E}{\partial x_k} \cdot w_{jk}.
\]

Uteži posodabljamo z osnovnim pravilom gradientnega spusta:
\[
\Delta w = -\epsilon \frac{\partial E}{\partial w},
\]
ali s faktorjem inercije:
\[
\Delta w(t) = -\epsilon \frac{\partial E}{\partial w}(t) + \alpha \Delta w(t-1),
\]
kjer je \( \epsilon \) hitrost učenja, \( \alpha \in [0,1] \) pa parameter, ki določa vpliv prejšnjih sprememb uteži.

Uteži na začetku inicializiramo naključno. Metoda povratnega razširjanja omogoča učinkovito učenje brez potrebe po drugem odvodu (Hessianu), kar je še posebej pomembno za izvedbo na realni strojni opremi.

Simetrija v arhitekturi mreže lahko vpliva na učenje; če so izhodne enote simetrične, morajo vmesne skrite enote omogočiti razločevanje vhodov. Zato so skrite enote ključnega pomena za uspešno predstavitev in razločevanje podatkov.

% THIS IS END OF NEW CHAPTER
