\chapter{Uvod}
Nevronske mreže so modeli strojnega učenja, ki zelo dobro prepoznavajo vzorce in rešujejo kompleksne probleme. Zaradi njihove uporabe v medicini, ekonomiji, avtonomnih sistemih in še mnogo več, kjer bi pri napaki modela lahko prišlo do hujših posledic, je pomembno da preverimo njegovo verodostojnost. Pri tem je pomembno, da se prepričamo, da pri učenju modela, kljub visoki natančnosti, ni prišlo do zlorabe, kjer bi se recimo odločitve sprejemale na podlagi nekih nepredvidenih spremenljivk v podatkih, ki zares nimajo nobene povezave s problemom. Tukaj govorimo o funkcionalnem razumevanju modela in ne o razumevanju ostalih nizkonovijskih algoritmov in konceptov, saj te že razumemo zelo dobro. Interpretacijo bomo razumeli kot preslikavo abstraktnega koncepta v domeno, ki jo lahko ljudlje razumemo.\cite{MONTAVON20181}

V diplomskem delu si bomo ogledali metodo iz članka \cite{Gabella_2021}, katere ideja temelji na tem, da bi s topologijo lahko zajeli strukturo, ki nastane v procesu učenja. Najprej bomo predstavili matematično ozadje in ugotovitve predstavili v zaključku.

Cilj interpretacije nevronskih mrež je da bi lahko ljudje bolj zaupali umetni inteligenci ter da bi zagotovili pošteno, etično in odgovorno uporabo te tehnologije.

