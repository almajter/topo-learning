%%%%%%%%%%%%%%%%%%%%%%%%%%%%%%%%%%%%%%%%
% datoteka diploma-FRI-vzorec.tex
%
%POZOR: ta verzija ne producira pdf datoteke v pdf/A formatu!!!
%namenjena je le za nalogo pri Diplomskem seminarju!
%
% vzorčna datoteka za pisanje diplomskega dela v formatu LaTeX
% na UL Fakulteti za računalništvo in informatiko
%
% na osnovi starejših verzij vkup spravil Franc Solina, maj 2021
% prvo verzijo je leta 2010 pripravil Gašper Fijavž
%
% za upravljanje z literaturo ta vezija uporablja BibLaTeX
%
% svetujemo uporabo Overleaf.com - na tej spletni implementaciji LaTeXa ta vzorec zagotovo pravilno deluje
%
\documentclass[a4paper,12pt,openright,mat1]{book}
%\documentclass[a4paper, 12pt, openright, draft]{book}  Nalogo preverite tudi z opcijo draft, ki pokaže, katere vrstice so predolge! Pozor, v draft opciji, se slike ne pokažejo!
 
\usepackage[utf8]{inputenc}   % omogoča uporabo slovenskih črk kodiranih v formatu UTF-8
\usepackage[slovene,english]{babel}    % naloži, med drugim, slovenske delilne vzorce
\usepackage[pdftex]{graphicx}  % omogoča vlaganje slik različnih formatov
\usepackage{fancyhdr}          % poskrbi, na primer, za glave strani
\usepackage{amssymb}           % dodatni matematični simboli
\usepackage{amsmath}           % eqref, npr.
\usepackage[pdftex, colorlinks=true,
						citecolor=black, filecolor=black, 
						linkcolor=black, urlcolor=black,
						pdfproducer={LaTeX}, pdfcreator={LaTeX}]{hyperref}
\usepackage{hyperxmp}
\usepackage{csquotes}

\usepackage{color}
\usepackage{soul}

\usepackage{comment}
\usepackage{float}

\usepackage{algorithm}
\usepackage{algpseudocode}

\usepackage{tikz}
\usetikzlibrary{positioning}

\usepackage[
backend=biber,
style=numeric,
sorting=nty,
]{biblatex}

\addbibresource{literatura.bib} %Imports bibliography file


%%%%%%%%%%%%%%%%%%%%%%%%%%%%%%%%%%%%%%%%
%	DIPLOMA INFO
%%%%%%%%%%%%%%%%%%%%%%%%%%%%%%%%%%%%%%%%
\newcommand{\ttitle}{Topologija učenja nevronskih mrež}
\newcommand{\ttitleEn}{Topology of learning in neural networks}
\newcommand{\tsubject}{\ttitle}
\newcommand{\tsubjectEn}{\ttitleEn}
\newcommand{\tauthor}{Maj Alter}
\newcommand{\tkeywords}{Nevronske mreže, Metoda glavnih komponent, Mapper algoritem}
\newcommand{\tkeywordsEn}{Neural networks, Principal component analysis, Mapper algorithm}

%%%%%%%%%%%%%%%%%%%%%%%%%%%%%%%%%%%%%%%%
%	HYPERREF SETUP
%%%%%%%%%%%%%%%%%%%%%%%%%%%%%%%%%%%%%%%%
\hypersetup{pdftitle={\ttitle}}
\hypersetup{pdfsubject=\ttitleEn}
\hypersetup{pdfauthor={\tauthor}}
\hypersetup{pdfkeywords=\tkeywordsEn}

%%%%%%%%%%%%%%%%%%%%%%%%%%%%%%%%%%%%%%%%
% postavitev strani
%%%%%%%%%%%%%%%%%%%%%%%%%%%%%%%%%%%%%%%%  

\addtolength{\marginparwidth}{-20pt} % robovi za tisk
\addtolength{\oddsidemargin}{40pt}
\addtolength{\evensidemargin}{-40pt}

\renewcommand{\baselinestretch}{1.3} % ustrezen razmik med vrsticami
\setlength{\headheight}{15pt}        % potreben prostor na vrhu
\renewcommand{\chaptermark}[1]%
{\markboth{\MakeUppercase{\thechapter.\ #1}}{}} \renewcommand{\sectionmark}[1]%
{\markright{\MakeUppercase{\thesection.\ #1}}} \renewcommand{\headrulewidth}{0.5pt} \renewcommand{\footrulewidth}{0pt}
\fancyhf{}
\fancyhead[LE,RO]{\sl \thepage} 
%\fancyhead[LO]{\sl \rightmark} \fancyhead[RE]{\sl \leftmark}
\fancyhead[RE]{\sc \tauthor}              % dodal Solina
\fancyhead[LO]{\sc Diplomska naloga}     % dodal Solina


\newcommand{\BibLaTeX}{{\sc Bib}\LaTeX}
\newcommand{\BibTeX}{{\sc Bib}\TeX}

%%%%%%%%%%%%%%%%%%%%%%%%%%%%%%%%%%%%%%%%
% naslovi
%%%%%%%%%%%%%%%%%%%%%%%%%%%%%%%%%%%%%%%%  

\newcommand{\autfont}{\Large}
\newcommand{\titfont}{\LARGE\bf}
\newcommand{\clearemptydoublepage}{\newpage{\pagestyle{empty}\cleardoublepage}}
\setcounter{tocdepth}{1}	      % globina kazala

%%%%%%%%%%%%%%%%%%%%%%%%%%%%%%%%%%%%%%%%
% konstrukti
%%%%%%%%%%%%%%%%%%%%%%%%%%%%%%%%%%%%%%%%  
\newtheorem{izrek}{Izrek}[chapter]
\newtheorem{trditev}{Trditev}[izrek]
\newtheorem{definicija}{Definicija}[izrek]
\newenvironment{dokaz}{\emph{Dokaz.}\ }{\hspace{\fill}{$\Box$}}


%%%%%%%%%%%%%%%%%%%%%%%%%%%%%%%%%%%%%%%%%%%%%%%%%%%%%%%%%%%%%%%%%%%%%%%%%%%%%%%
%% PDF-A
%%%%%%%%%%%%%%%%%%%%%%%%%%%%%%%%%%%%%%%%%%%%%%%%%%%%%%%%%%%%%%%%%%%%%%%%%%%%%%%

%%%%%%%%%%%%%%%%%%%%%%%%%%%%%%%%%%%%%%%% 
% define medatata
%%%%%%%%%%%%%%%%%%%%%%%%%%%%%%%%%%%%%%%% 
\def\Title{\ttitle}
\def\Author{\tauthor, ma1983@student.uni-lj.si}
\def\Subject{\ttitleEn}
\def\Keywords{\tkeywordsEn}

%%%%%%%%%%%%%%%%%%%%%%%%%%%%%%%%%%%%%%%% 
% \convertDate converts D:20080419103507+02'00' to 2008-04-19T10:35:07+02:00
%%%%%%%%%%%%%%%%%%%%%%%%%%%%%%%%%%%%%%%% 
\def\convertDate{%
    \getYear
}

{\catcode`\D=12
 \gdef\getYear D:#1#2#3#4{\edef\xYear{#1#2#3#4}\getMonth}
}
\def\getMonth#1#2{\edef\xMonth{#1#2}\getDay}
\def\getDay#1#2{\edef\xDay{#1#2}\getHour}
\def\getHour#1#2{\edef\xHour{#1#2}\getMin}
\def\getMin#1#2{\edef\xMin{#1#2}\getSec}
\def\getSec#1#2{\edef\xSec{#1#2}\getTZh}
\def\getTZh +#1#2{\edef\xTZh{#1#2}\getTZm}
\def\getTZm '#1#2'{%
    \edef\xTZm{#1#2}%
    \edef\convDate{\xYear-\xMonth-\xDay T\xHour:\xMin:\xSec+\xTZh:\xTZm}%
}

%\expandafter\convertDate\pdfcreationdate 

%%%%%%%%%%%%%%%%%%%%%%%%%%%%%%%%%%%%%%%%
% get pdftex version string
%%%%%%%%%%%%%%%%%%%%%%%%%%%%%%%%%%%%%%%% 
\newcount\countA
\countA=\pdftexversion
\advance \countA by -100
\def\pdftexVersionStr{pdfTeX-1.\the\countA.\pdftexrevision}


%%%%%%%%%%%%%%%%%%%%%%%%%%%%%%%%%%%%%%%%
% XMP data
%%%%%%%%%%%%%%%%%%%%%%%%%%%%%%%%%%%%%%%%  
\usepackage{xmpincl}
%\includexmp{pdfa-1b}

%%%%%%%%%%%%%%%%%%%%%%%%%%%%%%%%%%%%%%%%
% pdfInfo
%%%%%%%%%%%%%%%%%%%%%%%%%%%%%%%%%%%%%%%%  
\pdfinfo{%
    /Title    (\ttitle)
    /Author   (\tauthor, ma1983@student.uni-lj.si)
    /Subject  (\ttitleEn)
    /Keywords (\tkeywordsEn)
    /ModDate  (\pdfcreationdate)
    /Trapped  /False
}

%%%%%%%%%%%%%%%%%%%%%%%%%%%%%%%%%%%%%%%%
% znaki za copyright stran
%%%%%%%%%%%%%%%%%%%%%%%%%%%%%%%%%%%%%%%%  

\newcommand{\CcImageCc}[1]{%
	\includegraphics[scale=#1]{resources/cc/cc_cc_30.pdf}%
}
\newcommand{\CcImageBy}[1]{%
	\includegraphics[scale=#1]{resources/cc/cc_by_30.pdf}%
}
\newcommand{\CcImageSa}[1]{%
	\includegraphics[scale=#1]{resources/cc/cc_sa_30.pdf}%
}

%%%%%%%%%%%%%%%%%%%%%%%%%%%%%%%%%%%%%%%%%%%%%%%%%%%%%%%%%%%%%%%%%%%%%%%%%%%%%%%
%%%%%%%%%%%%%%%%%%%%%%%%%%%%%%%%%%%%%%%%%%%%%%%%%%%%%%%%%%%%%%%%%%%%%%%%%%%%%%%

\begin{document}
\selectlanguage{slovene}
\frontmatter
\setcounter{page}{1} %
\renewcommand{\thepage}{}       % preprečimo težave s številkami strani v kazalu

%%%%%%%%%%%%%%%%%%%%%%%%%%%%%%%%%%%%%%%%
%naslovnica
%%%%%%%%%%%%%%%%%%%%%%%%%%%%%%%%%%%%%%%%
 \thispagestyle{empty}%
   \begin{center}
    {\large\sc Univerza v Ljubljani\\%
%      Fakulteta za elektrotehniko\\% za študijski program Multimedija
%      Fakulteta za upravo\\% za študijski program Upravna informatika
      Fakulteta za računalništvo in informatiko\\%
      Fakulteta za matematiko in fiziko\\% za študijski program Računalništvo in matematika
     }
    \vskip 10em%
    {\autfont \tauthor\par}%
    {\titfont \ttitle \par}%
    {\vskip 3em \textsc{DIPLOMSKO DELO\\[5mm]         % dodal Solina za ostale študijske programe
%    VISOKOŠOLSKI STROKOVNI ŠTUDIJSKI PROGRAM\\ PRVE STOPNJE\\ RAČUNALNIŠTVO IN INFORMATIKA}\par}%
%     UNIVERZITETNI  ŠTUDIJSKI PROGRAM\\ PRVE STOPNJE\\ RAČUNALNIŠTVO IN INFORMATIKA}\par}%
%    INTERDISCIPLINARNI UNIVERZITETNI\\ ŠTUDIJSKI PROGRAM PRVE STOPNJE\\ MULTIMEDIJA}\par}%
%    INTERDISCIPLINARNI UNIVERZITETNI\\ ŠTUDIJSKI PROGRAM PRVE STOPNJE\\ UPRAVNA INFORMATIKA}\par}%
    INTERDISCIPLINARNI UNIVERZITETNI\\ ŠTUDIJSKI PROGRAM PRVE STOPNJE\\ RAČUNALNIŠTVO IN MATEMATIKA}\par}%
    \vfill\null%
% izberite pravi habilitacijski naziv mentorja!
    {\large \textsc{Mentor}: doc. dr. Dejan Govc\par}%
%   {\large \textsc{Somentor}:  viš. pred./doc./izr. prof./prof. dr.  Martin Krpan \par}%
    {\vskip 2em \large Ljubljana, \the\year \par}%
\end{center}
% prazna stran
%\clearemptydoublepage      
% izjava o licencah itd. se izpiše na hrbtni strani naslovnice

%%%%%%%%%%%%%%%%%%%%%%%%%%%%%%%%%%%%%%%%
%copyright stran
%%%%%%%%%%%%%%%%%%%%%%%%%%%%%%%%%%%%%%%%
\newpage
\thispagestyle{empty}

\vspace*{5cm}
{\small \noindent
To delo je ponujeno pod licenco \textit{Creative Commons Priznanje avtorstva-Deljenje pod enakimi pogoji 2.5 Slovenija} (ali novej\v so razli\v cico).
To pomeni, da se tako besedilo, slike, grafi in druge sestavine dela kot tudi rezultati diplomskega dela lahko prosto distribuirajo,
reproducirajo, uporabljajo, priobčujejo javnosti in predelujejo, pod pogojem, da se jasno in vidno navede avtorja in naslov tega
dela in da se v primeru spremembe, preoblikovanja ali uporabe tega dela v svojem delu, lahko distribuira predelava le pod
licenco, ki je enaka tej.
Podrobnosti licence so dostopne na spletni strani \href{http://creativecommons.si}{creativecommons.si} ali na Inštitutu za
intelektualno lastnino, Streliška 1, 1000 Ljubljana.

\vspace*{1cm}
\begin{center}% 0.66 / 0.89 = 0.741573033707865
\CcImageCc{0.741573033707865}\hspace*{1ex}\CcImageBy{1}\hspace*{1ex}\CcImageSa{1}%
\end{center}
}

\vspace*{1cm}
{\small \noindent
Izvorna koda diplomskega dela, njeni rezultati in v ta namen razvita programska oprema je ponujena pod licenco GNU General Public License,
različica 3 (ali novejša). To pomeni, da se lahko prosto distribuira in/ali predeluje pod njenimi pogoji.
Podrobnosti licence so dostopne na spletni strani \url{http://www.gnu.org/licenses/}.
}

\vfill
\begin{center} 
\ \\ \vfill
{\em
Besedilo je oblikovano z urejevalnikom besedil \LaTeX.}
\end{center}

% prazna stran
\clearemptydoublepage

%%%%%%%%%%%%%%%%%%%%%%%%%%%%%%%%%%%%%%%%
% stran 3 med uvodnimi listi
%%%%%%%%%%%%%%%%%%%%%%%%%%%%%%%%%%%%%%%%
\thispagestyle{empty}
\
\vfill

\bigskip
\noindent\textbf{Kandidat:} Maj Alter\\
\noindent\textbf{Naslov:} Topologija učenja nevronskih mrež\\
% vstavite ustrezen naziv študijskega programa!
\noindent\textbf{Vrsta naloge:} Diplomska naloga na interdisciplinarnem univerzitetnem študijskem programu prve stopnje Računalništvo in matematika \\
% izberite pravi habilitacijski naziv mentorja!
\noindent\textbf{Mentor:} doc. dr. Dejan Govc\\
%\noindent\textbf{Somentor:} isto kot za mentorja

\bigskip
\noindent\textbf{Opis:}\\
TODO: Besedilo teme diplomskega dela študent prepiše iz študijskega informacijskega sistema, kamor ga je vnesel mentor. 
V nekaj stavkih bo opisal, kaj pričakuje od kandidatovega diplomskega dela. 
Kaj so cilji, kakšne metode naj uporabi, morda bo zapisal tudi ključno literaturo.

\bigskip
\noindent\textbf{Title:} Topologija učenja nevronskih mrež

\bigskip
\noindent\textbf{Description:}\\
TODO: opis diplome v angleščini

\vfill



\vspace{2cm}

% prazna stran
\clearemptydoublepage

%%%%%%%%%%%%%%%%%%%%%%%%%%%%%%%%%%%%%%%%
% zahvala
%%%%%%%%%%%%%%%%%%%%%%%%%%%%%%%%%%%%%%%%
\thispagestyle{empty}\mbox{}\vfill\null\it%
\noindent
Na tem mestu bi se rad zahvalil družini za podporo v času mojega študija in mentorju doc. dr. Dejanu Govcu za vso pomoč pri pisanju diplomskega dela.
\rm\normalfont

% prazna stran
\clearemptydoublepage

% prazna stran
\clearemptydoublepage


%%%%%%%%%%%%%%%%%%%%%%%%%%%%%%%%%%%%%%%%
% kazalo
%%%%%%%%%%%%%%%%%%%%%%%%%%%%%%%%%%%%%%%%
\pagestyle{empty}
\def\thepage{}% preprečimo težave s številkami strani v kazalu
\tableofcontents{}


% prazna stran
\clearemptydoublepage

%%%%%%%%%%%%%%%%%%%%%%%%%%%%%%%%%%%%%%%%
% seznam kratic
%%%%%%%%%%%%%%%%%%%%%%%%%%%%%%%%%%%%%%%%
\chapter*{Seznam uporabljenih kratic}

\noindent\begin{tabular}{p{0.11\textwidth}|p{.39\textwidth}|p{.39\textwidth}}    % po potrebi razširi prvo kolono tabele na račun drugih dveh!
  {\bf kratica} & {\bf angleško}                              & {\bf slovensko} \\ \hline
  {\bf TDA}      & topological data analysis               & topološka analiza podatkov \\
  {\bf PCA}      & principal component analysis               & metoda glavnih komponent \\

 % {\bf DBMS} & database management system & sistem za upravljanje podatkovnih baz \\
 % {\bf SVM}   & support vector machine              & metoda podpornih vektorjev \\
%  \dots & \dots & \dots \\
\end{tabular}


% prazna stran
\clearemptydoublepage

%%%%%%%%%%%%%%%%%%%%%%%%%%%%%%%%%%%%%%%%
% povzetek
%%%%%%%%%%%%%%%%%%%%%%%%%%%%%%%%%%%%%%%%
\addcontentsline{toc}{chapter}{Povzetek}
\chapter*{Povzetek}

\noindent\textbf{Naslov:} \ttitle
\bigskip

\noindent\textbf{Avtor:} \tauthor
\bigskip

%\noindent\textbf{Povzetek:} 
\noindent Z nevronskimi mrežami lahko rešimo širok nabor problemov, kljub temu pa mnogokrat težko interpretiramo, kaj se ti modeli zares naučijo. Uteži v nevronski mreži se v postopku učenja razvijejo tako, da so zadani problem sposobne rešiti. V diplomskem delu bomo spoznali metodo, s katero lahko s pomočjo topološke analize podatkov zajamemo strukturo sestavljeno iz uteži, ki nastane med učenjem. Delo temelji na članku \cite{Gabella_2021}, kjer je metoda opisana in predstavljena na preprostem primeru. Opisali bomo matematično ozadje in koncepte s pomočjo katerih lahko na podlagi algoritma Mapper, grafično interpretiramo nevronsko mrežo.  \\ \\


\bigskip

\noindent\textbf{Ključne besede:} \tkeywords.
% prazna stran
\clearemptydoublepage

%%%%%%%%%%%%%%%%%%%%%%%%%%%%%%%%%%%%%%%%
% abstract
%%%%%%%%%%%%%%%%%%%%%%%%%%%%%%%%%%%%%%%%
\selectlanguage{english}
\addcontentsline{toc}{chapter}{Abstract}
\chapter*{Abstract}

\noindent\textbf{Title:} \ttitleEn
\bigskip

\noindent\textbf{Author:} \tauthor
\bigskip

%\noindent\textbf{Abstract:} 
\noindent With neural networks, we can solve a wide range of problems, yet it is often difficult to interpret what these models actually learn. The weights in a neural network develop during the learning process to solve the given problem. In this thesis, we will explore a method by which we can use topological data analysis to capture the structure composed of weights that arises during learning. The work is based on the article [1], where the method is described and demonstrated on a simple example. We will describe the mathematical background and concepts that allow us to graphically interpret the neural network based on the Mapper algorithm.
\bigskip

\noindent\textbf{Keywords:} \tkeywordsEn.
\selectlanguage{slovene}
% prazna stran
\clearemptydoublepage

%%%%%%%%%%%%%%%%%%%%%%%%%%%%%%%%%%%%%%%%
% vsebina
%%%%%%%%%%%%%%%%%%%%%%%%%%%%%%%%%%%%%%%%
\mainmatter
\setcounter{page}{1}
\pagestyle{fancy}

%\chapter{Uvod}
\chapter{Uvod}

Nevronske mreže so modeli strojnega učenja, ki zelo dobro prepoznavajo vzorce in rešujejo kompleksne probleme. Zaradi njihove uporabe v medicini, ekonomiji, avtonomnih sistemih in še mnogo več, kjer bi pri napaki modela lahko prišlo do hujših posledic, je pomembno da preverimo njegovo verodostojnost. Pri tem je pomembno, da se prepričamo, da pri učenju modela, kljub visoki natančnosti, ni prišlo do zlorabe, kjer bi se recimo odločitve sprejemale na podlagi nekih nepredvidenih spremenljivk v podatkih, ki zares nimajo nobene povezave s problemom. Tukaj govorimo o funkcionalnem razumevanju modela in ne o razumevanju ostalih nizkonovijskih algoritmov in konceptov, saj te že razumemo zelo dobro. Interpretacijo bomo razumeli kot preslikavo abstraktnega koncepta v domeno, ki jo lahko ljudlje razumemo.\cite{MONTAVON20181}

V diplomskem delu si bomo ogledali metodo iz članka~\cite{Gabella2021}, katere ideja temelji na tem, da bi s topologijo lahko zajeli strukturo, ki nastane v procesu učenja. Najprej bomo predstavili matematično ozadje in ugotovitve predstavili v zaključku.

Cilj interpretacije nevronskih mrež je da bi lahko ljudje bolj zaupali umetni inteligenci ter da bi zagotovili pošteno, etično in odgovorno uporabo te tehnologije.



\include{poglavja/nevronske_mreze}

\chapter{Metoda glavnih komponent}
Vektorji vhodnih uteži na sloju $N^{(i)}$ so velikosti $N^{(i - 1)}$. Te številke so običajno veliko večje od 2 ali 3, zato si teh vektorjev ne moremo predstavljati, kot točk v 2 oz. 3 dimenzionalnem prostoru. V tem poglavju si bomo pogledali linearno metodo za zmanjševanje dimenzij, ki se imenuje metoda glavnih komponent. Pri tem bomo sledili \cite{shlens2014tutorial}.

\section{Sprememba baze}
Cilj metode glavnih komponent je zbirko podatkov predstaviti z bazo, katere bazni vektorji bodo po vrsti razkrili največ informacije o originalnih podatkih. Glavna ideja je, da bodo najpomembnejši bazni vektorji zadostovali za predstavo strukture podatkov, medtem ko bodo manj pomembnejši le filtrirali morebitne šume. Originalne podatke, ki smo jih predstavili z matriko $X$, bomo torej transformirali v matriko $Y$, ki bo predstavljala podatke zapisane v novi bazi. Vsak nov bazni vektor lahko zapišemo kot linearno kombinacijo originalnih baznih vektorjev, zato lahko poiščemo matriko linearne preslikave $P$, tako da bo veljalo:
\begin{equation}
    Y^T = PX^T
\end{equation}

\section{Napaka rekonstrukcije in varianca}
Vrednost podatkov v novi bazi dobimo s pravokotno projekcijo na nove normirane bazne vektorje. Iskanja nove komponente se lahko lotimo, tako da poiščemo enotski vektor $u_1$, ki bo po projekciji podatkov minimiziral napako rekonstrukcije originalnega prostora.
\begin{equation}
E(u_1) = \sum_{i} \left\lVert x^{(i)} - (u_{1}^{T}{x}^{(i)})u_{1} \right\rVert^2
\end{equation}

\begin{figure}[H]
    \centering
    \includegraphics[width=0.7\linewidth]{resources/napakaRekonstrukcije.png}
    \caption{Napaka rekonstrukcije. Vir: \cite{dataset}}
    \label{fig:backprop}
\end{figure}
\begin{comment}
    https://www.google.com/url?sa=i&url=https%3A%2F%2Fstats.stackexchange.com%2Fquestions%2F194278%2Fmeaning-of-reconstruction-error-in-pca-and-lda&psig=AOvVaw2fogYikUFLgwt2PUiiO5BI&ust=1714336714385000&source=images&cd=vfe&opi=89978449&ved=0CBIQjRxqFwoTCNDVwoig44UDFQAAAAAdAAAAABAE
\end{comment}

To pa je ravno ekvivalenten problem iskanja vektorja, ki bo po projekciji originalnih podatkov jih ohranjal najbolj razpršene. Razpršenost merimo z varianco, ki je povprečna oddaljenost projekcije podatkov od projekcije središčne točke podatkov $\bar{x} = \frac{1}{m} \sum_{i} x^{(i)}$:
\begin{equation}
    \mathrm{Var}(u_1^{\top}X^{\top}) = \frac{1}{m} \sum_{i=1}^{m} \left(u_1^{\top}x^{(i)} - u_1^{\top}\bar{x}\right)^2
\end{equation}
\begin{figure}[H]
    \centering
    \includegraphics[width=1\linewidth]{resources/pcaDvaPogleda.png}
    \caption{Varianca in napaka rekonstrukcije. Vir: \cite{quora_pca_explanation}}
    \label{fig:backprop}
\end{figure}
\begin{comment}
    https://alexhwilliams.info/itsneuronalblog/img/pca/pca_two_views.png
\end{comment}
Dokažimo zgornjo trditev.
\begin{trditev}
    $u_{1} = \underset{u:\|\mathbf{u}\|_2=1}{\operatorname{argmin}}\ E(u) = \underset{u:\|\mathbf{u}\|_2=1}{\operatorname{argmax}}\ \operatorname{Var}(u^{\top}X^{\top})$
\end{trditev}
\begin{dokaz}
    \begin{equation}
\mathbf{u_{1}} = \underset{\mathbf{u}:\|\mathbf{u}\|_2=1}{\mathrm{argmin}} \frac{1}{m} \sum_{i=1}^{m} \left\| \mathbf{u}^{(i)} - (\mathbf{u}^T \mathbf{x}^{(i)})\mathbf{u} \right\|^2
\end{equation}
\begin{equation}
= \underset{\mathbf{u}:\|\mathbf{u}\|_2=1}{\mathrm{argmin}} \frac{1}{m} \sum_{i=1}^{m} \left\| \mathbf{x}^{(i)} \right\|^2 - (\mathbf{u}^T \mathbf{x}^{(i)})^2
\end{equation}
\begin{equation}
= \underset{\mathbf{u}:\|\mathbf{u}\|_2=1}{\mathrm{argmax}} \frac{1}{m} \sum_{i=1}^{m} (\mathbf{u}^T \mathbf{x}^{(i)})^2
\end{equation}
\end{dokaz} \cite{gormley2018pca} \\
\begin{comment}
    http://www.cs.cmu.edu/~mgormley/courses/606-607-f18/slides606/lecture11-pca.pdf
\end{comment}
Če enačbo za varianco malenkost preoblikujemo, dobimo naslednje. \cite{zupan2024uozp}
\begin{equation}
\begin{aligned}
\mathrm{Var}(u_1^TX) &= \frac{1}{m}\sum_{i=1}^{m}(u_1^Tx^{(i)} - u_1^T\bar{x})^2 \\
&= \frac{1}{m}\sum_{i=1}^{m}\left((u_1^Tx^{(i)})^2 - 2u_1^Tx^{(i)}u_1^T\bar{x} + (u_1^T\bar{x})^2\right) \\
&= \frac{1}{m}\sum_{i=1}^{m}u_1^T\left(x^{(i)}(x^{(i)})^T - 2x^{(i)}\bar{x}^T + \bar{x}\bar{x}^T\right)u_1 \\
&= u^T \underbrace{\left( \frac{1}{m} \sum_{i=1}^{m} (x_i - \mu)(x_i - \mu)^T \right)}_{\text{Kovariančna matrika } C} u
\end{aligned}
\end{equation}
Zgornjo maksimizacijo variance lahko pod pogojem, da je $u_1$ enotski vektor, rešimo s pomočjo Langrangeovih multiplikatorjev.
\begin{equation}
f(u_1) = u_1^T C u_1 + \lambda_1 (1 - u_1^T u_1)
\end{equation}
\begin{equation}
\frac{\partial f(u_1)}{\partial u_1} = C u_1 - \lambda_1 u_1 = 0
\end{equation}

Po preoblikovanju enačbe, sledi da so to rešitve ravno lastni vektorji kovariančne matrike.
\begin{equation}
C u_1 = \lambda_1 u_1
\end{equation}
Če zgornjo še malo preoblikujemo, ugotovimo, da lastni vektor z največjo lastno vrednostjo ohranja največ informacije o originalnem prostoru.
\begin{equation}
    u_1^T C u_1 = u_1^T \lambda_1 u_1 = \lambda_1 u_1^T u_1 = \lambda_1
\end{equation}
Podobno lahko za ostale nove bazne vektorje izberemo ostale lastne vektorje kovariančne matrike z najvišjimi lastnimi vrednosti.

Opazimo lahko tudi, da je kovariančna matrika simetrična $\Sigma = \Sigma^T$, zato bodo lastni vektorji ortogonalni, torej preslikava, ki smo jo dobili je samo rotacija originalnega prostora.

\section{Rešitev z uporabo SVD}
Lastne vektorje kovariančne matrike C lahko izračunamo s pomočjo singularnega razcepa. \cite{zupan2024uozp}
\begin{equation}
    C = U \Sigma V^T
\end{equation}
Nato prvih k lastnih vektorjev vstavimo v matriko
\begin{equation}
    P = [u_1, u_2, \dots, u_k]
\end{equation}
Z matriko $P$, lahko sedaj preslikamo originalne podatke v nižje k-dimenzionalni prostor. 
\begin{equation}
y_i = P^T (x_i - \mu)
\end{equation}  

\chapter{Topološka analiza podatkov}

Topološka analiza podatkov (angl. Topological Data Analysis, TDA) je sodoben pristop k analizi podatkov, s katerim poskušamo razumeti obliko visokodimenzionalnih podatkovnih množic. Preučuje, kako so podatki med seboj povezani in razporejeni v prostoru, ali obstajajo kakšne povezave, zanke, praznine ter iz koliko komponent (gruč) so sestavljeni.
V praksi so se metode TDA izkazale kot zelo učinkovite pri analizi visokodimenzionalnih, razpršenih podatkov, ki vsebujejo veliko šuma. Ena izmed izmed takih metod algoritem Mapper, ki podatkovno množico pretvori v preprost simplicialni kompleks.

\section{Simpleksi}

Simpleksi so osnovni gradniki simplicialnih kompleksov. Intuitivno so $n$-simpleksi geometrijski objekti z $(n + 1)$ oglišči, ki ležijo v $n$-dimenzionalnem prostoru in jih ne moremo vstaviti v nižje dimenzionalni prostor. Za gradnjo $n$-simpleksa v $n$-dimenzionalnem prostoru potrebujemo $(n + 1)$ afino neodvisnih točk.

\begin{definicija}
    Množica točk $\{x_0, \dots, x_n\}$ v vektorskem prostoru $V$ je \textbf{afino neodvisna}, če je $\{x_1 - x_0, \dots, x_n - x_0\}$ linearno neodvisna.
\end{definicija}

S pomočjo zgornje definicije lahko simplekse matematično definiramo na naslednji način~\cite{Munkers84}.

\begin{definicija}
Naj bo \(X = \{x_0, x_1, \dots, x_n\}\) množica afino neodvisnih točk v prostoru \(\mathbb{R}^d\), kjer velja \(n \leq d\). Konveksno ovojnico množice \(X\),
\[
\operatorname{conv}(X) = \left\{ \sum_{i=0}^n \alpha_i x_i \;\middle|\; \alpha_i \in [0,1], \sum_{i=0}^n \alpha_i = 1 \right\},
\]
imenujemo \(n\)-simpleks na množici \(X\).
\end{definicija}

\begin{figure}[H]
  \centering
  \begin{tikzpicture}[scale=1.2, every node/.style={font=\small}]

  % 0-simplex
  \fill[myblue!60] (0,0) circle (2pt);
  \node[below=4pt] at (0,-0.2) {0 - simpleks};

  % 1-simplex
  \fill[myblue!60] (2,0) circle (2pt);
  \fill[myblue!60] (3,0) circle (2pt);
  \draw[thick] (2,0) -- (3,0);
  \node[below=4pt] at (2.5,-0.2) {1 - simpleks};

  % 2-simplex
  \coordinate (A) at (5.5,0);
  \coordinate (B) at (6.5,0);
  \coordinate (C) at (6,1.2);
  \fill[myblue!10] (A) -- (B) -- (C) -- cycle;
  \draw[thick] (A) -- (B) -- (C) -- cycle;
  \foreach \p in {A,B,C}
    \fill[myblue!60] (\p) circle (2pt);
  \node[below=4pt] at (6,-0.2) {2 - simpleks};

  % 3-simplex
  \coordinate (D) at (9,0);
  \coordinate (E) at (10,0);
  \coordinate (F) at (9.5,1.2);
  \coordinate (G) at (9.5,0.5);
  \fill[myblue!10] (D) -- (E) -- (F) -- cycle;
  \draw[thick] (D) -- (E) -- (F) -- cycle;
  \draw[thick] (D) -- (G) -- (E);
  \draw[thick] (F) -- (G);
  \draw[thick] (D) -- (F);
  \draw[thick] (E) -- (F);
  \foreach \p in {D,E,F,G}
    \fill[myblue!60] (\p) circle (2pt);
  \node[below=4pt] at (9.5,-0.2) {3 - simpleks};

\end{tikzpicture}

  \caption{Simpleksi v dimenzijah 0, 1, 2 in 3}~\label{fig:basic-simplices}
\end{figure}

%Točke \(X\) imenujemo oglišča \(\sigma\), simplekse ki jih razpenjajo podmnožice \(X\) pa imenujemo lica \(\sigma\).

\section{Simplicialni kompleksi}
Simplicialni kompleksi so kombinatorični opisi topološkega prostora.
\begin{definicija}
Simplicialni kompleks \(K\) v prostoru \(R^d\) je množica simpleksov, tako da velja:
  \begin{enumerate}
    \item Vsako lice poljubnega simpleksa v simplicialnemu kompleksu \(K\) je tudi samo simpleks v \(K\).
    \item Naj bosta \(\sigma_1\) in \(\sigma_2\) poljubna simpleksa v \(K\). \v{C}e se sekata, potem je njun presek \(\sigma_1 \cap \sigma_2\) lice simpleksov \(\sigma_1\) in \(\sigma_2\) ter je simpleks v \(K\).
  \end{enumerate}
\end{definicija}

\begin{figure}[H]
  \centering
  \begin{tikzpicture}[scale=2, vertex/.style={circle, fill=black, inner sep=1.5pt}]

% === Tetrahedron faces ===
\fill[myblue!30] (1,0) -- (0.5,0.9) -- (0.6,-0.7) -- cycle;        % right face
\fill[myblue!30] (0,0) -- (1,0) -- (0.6,-0.7) -- cycle;             % bottom face
\fill[myblue!50] (0,0) -- (0.6,-0.7) -- (0.5,0.9) -- cycle;         % left face

  % === Fill the triangle on the right ===
  \fill[myblue!30] (2.4,-0.2) -- (2.9,0.4) -- (2.2,0.5) -- cycle;

  % === Tetrahedron edges ===
  \draw[dashed, myblue] (0,0) -- (1,0);              % base front edge
  \draw[thick, myblue] (0,0) -- (0.5,0.9);          % left front edge
  \draw[thick, myblue] (1,0) -- (0.5,0.9);          % right front edge
  \draw[thick, myblue] (0,0) -- (0.6,-0.7);         % left bottom edge
  \draw[thick, myblue] (0.6,-0.7) -- (0.5,0.9);     % back-left edge
  \draw[thick, myblue] (1,0) -- (0.6,-0.7);        % back-bottom hidden edge

  % === Edge between components ===
  \draw[thick, myblue] (1,0) -- (1.8,0.1);

  % === Triangle on the right ===
  \draw[thick, myblue] (2.4,-0.2) -- (2.9,0.4) -- (2.2,0.5) -- cycle;

  % === Vertical edge above the triangle ===
  \draw[thick, myblue] (2.55,0.95) -- (2.2,0.45);

  % === Vertices ===
  \foreach \x/\y in {
    0/0, 1/0, 0.5/0.9, 0.6/-0.7,
    1.8/0.1,
    2.4/-0.2, 2.9/0.4, 2.2/0.5,
    2.55/0.95
  } {
    \node[vertex] at (\x,\y) {};
  }

\end{tikzpicture}

  \caption{Simplicialni kompleks}~\label{fig:simplicial-complex}
\end{figure}

\section{Vietoris–Ripsova filtracija}
Eden najbolj uporabljenih načinov za gradnjo simplicialnih kompleksov iz podatkov je \textbf{Vietoris–Ripsov kompleks}. Postopek je sledeč:
\begin{itemize}
  \item Imamo množico točk v prostoru in izberemo parameter $\epsilon > 0$.
  \item Povežemo vsaka dva podatkovna vektorja (točki), ki sta bližje kot $\epsilon$.
  \item Če so vse povezave med trojico točk prisotne, dodamo trikotnik.
  \item Če vse povezave med četverico točk obstajajo, dodamo tetraeder itd.
\end{itemize}

S tem postopkom zgradimo simplicialni kompleks za določen $\epsilon$. Če ta parameter povečujemo, dobimo zaporedje kompleksov:
\[
K_0 \subseteq K_1 \subseteq K_2 \subseteq \cdots \subseteq K_n,
\]
kar imenujemo \textbf{filtracija}. S filtracijo opazujemo, kako se topološka struktura spreminja skozi različne vrednosti $\epsilon$.

\section{Izrek o živcu}

Na naslednji način lahko na končnem pokritju definiramo simplicialni kompleks, ki ga imenujemo živec pokritja

\begin{definicija}
Naj bo $U = \{U_\alpha\}_{\alpha \in A}$ končno pokritje prostora $X$. Živec pokritja $U$ definiramo kot simplicialni kompleks $\mathcal{N}(U)$, katerega množica oglišč je indeksna množica $A$, in kjer je družina $\{\alpha_0, \alpha_1, \dots, \alpha_k\}$ $k$-simpleks v $\mathcal{N}(U)$ natanko tedaj, ko velja $U_{\alpha_0} \cap U_{\alpha_1} \cap \dots \cap U_{\alpha_k} \ne \emptyset$.
\end{definicija}

V splošnem ne moremo trditi, da bo živec imel enake topološke lastnosti kot prostor \( X \). Ravno zato je \textit{izrek o živcu} ključen: pod določenimi pogoji zagotavlja, da sta živec \( \mathcal{N}(\mathcal{U}) \) in prostor \( X \) homotopsko ekvivalentna, tj. jih lahko zvezno preoblikujemo drug v drugega.

\begin{izrek}[Nervni izrek, Hatcher Cor. 4G.3]
Naj bo \(X\) parakomptni Hausdorffov prostor in \(\mathcal{U} = \{U_i\}_{i\in I}\) odprto pokritje prostora \(X\), v katerem so vse končne nepravilne preseke \(U_{i_0}\cap\dots\cap U_{i_k}\) kontraktibilni (tj. cover je \emph{dobar}). Potem je živec \(\mathcal{N}(\mathcal{U})\) homotopsko ekvivalenten prostoru \(X\):
\[
X \simeq \lvert\mathcal{N}(\mathcal{U})\rvert.
\]
\end{izrek}

\begin{dokaz}[Oskrbljen skica dokaz]
1. Ker je \(X\) parakomptni Hausdorffov, obstaja subtendentna \emph{particija enotnosti} \(\{\varphi_i: X\to[0,1]\}\), kjer \(\varphi_i(x)=0\) zunaj \(U_i\), in \(\sum_i\varphi_i(x)=1\) za vsak \(x\).

2. Definiramo preslikavo
\[
f: X \;\longrightarrow\; \lvert\mathcal{N}(\mathcal{U})\rvert,  
\qquad
x \;\longmapsto\; \sum_i \varphi_i(x)\,v_i,
\]
kjer so \(v_i\) oglišča ugnezdene kompleksa, pomešana s powersko kombinacijo. Zaradi kontraktibilnih presekov ta preslikava dejansko obišče samo predele, kjer je baricentričen sodoben angažma možen.

3. Hatcher dokaže, da je \(f\) homotopska ekvivalenca. Naredi jo tako, da pokaže komutativne verižnice in nato uporabi standardna topološka dejstva o homotopski ekvivalenci na pasulji.

4. Zaključimo, da je \(X\) homotopsko ekvivalenten \(\lvert\mathcal{N}(\mathcal{U})\rvert\).

Ta izrek je tako temeljna povezava med pokritjem prostora in simplicialnim povzetkom oblike.
\end{dokaz}


\section{Vztrajna homologija}
Vztrajna homologija je metoda, ki sledi, kako se topološke značilnosti (komponente, zanke, votline) pojavljajo in izginjajo skozi filtracijo.

Za vsako dimenzijo $k$ štejemo $k$-dimenzionalne luknje:
\begin{itemize}
  \item V dimenziji 0: koliko ločenih komponent obstaja.
  \item V dimenziji 1: koliko zank ali obročev.
  \item V dimenziji 2: koliko votlih prostorov.
\end{itemize}

Ko povečujemo $\epsilon$, se nekatere značilnosti pojavijo (rojstvo), druge izginejo (smrt). Zabeležimo jih v obliki:
\[
(\epsilon_\text{rojstvo}, \epsilon_\text{smrt})
\]

Daljši kot je razmik med rojstvom in smrtjo, bolj \textbf{vztrajna} je značilnost. Takšne značilnosti veljajo za pomembne. Te podatke vizualiziramo kot:
\begin{itemize}
  \item \textbf{Vztrajni diagram} – točke nad diagonalo $(\epsilon_\text{rojstvo}, \epsilon_\text{smrt})$.
  \item \textbf{Črtne kode} – vsaka značilnost je predstavljena kot daljica.
\end{itemize}

Persistentna homologija nam omogoča izluščiti robustne strukturne značilnosti podatkov in jih ločiti od šuma. Skupaj z Vietoris–Ripsovo filtracijo tvori eno osrednjih orodij v TDA.

\section{Mapper}
Primer v praksi: Odmeven primer uspešne uporabe TDA prihaja s področja biomedicine, kjer je topološka analiza razkrila pomemben nov vzorec v genomskih podatkih raka dojk. Nicolau s sodelavci so leta 2011 uporabili Mapper (v okviru metode Progression Analysis of Disease) za vizualizacijo transkripcijskih mikroarray podatkov tumorjev dojk
pmc.ncbi.nlm.nih.gov
. Rezultat je bil graf s tremi „vejnami“, ki predstavljajo različne podtipe raka – med njimi so avtorji odkrili novo podskupino znotraj estrogensko pozitivnih tumorjev (ER+), označeno z izrazito povišano ekspresijo gena c-MYB in znižano ekspresijo skupine vnetnih genov
pmc.ncbi.nlm.nih.gov
. Presenetljivo je, da so imele pacientke v tej skupini 100 % stopnjo preživetja in praktično brez metastaz, kar kaže na izjemno nenevaren potek bolezni
pmc.ncbi.nlm.nih.gov
. Analiza je to skupino izluščila povsem iz vzorcev genske ekspresije, brez uporabe kakršnih koli vnaprejšnjih kliničnih informacij (npr. podatkov o izidu bolezni)
pmc.ncbi.nlm.nih.gov
. Pomembno je poudariti, da je bila ta podskupina pred uporabo TDA nevidna za klasično gručenje – pri običajni grozdni analizi so bili namreč ti tumorji razpršeni med več različnih gruč, zaradi česar njihov skupni signal ni izstopal
pmc.ncbi.nlm.nih.gov
. Topološka analiza pa je razkrila, da ti tumorji tvorijo ločeno, izjemno homogeno skupino z jasnim molekularnim podpisom, ki ne ustreza nobenemu od prej znanih podtipov (ni šlo ne za luminalni A/B ne za t. i. normal-like rak)
pmc.ncbi.nlm.nih.gov
. Novo odkrito skupino so po glavni značilnosti poimenovali c-MYB+ podtip raka dojk
pmc.ncbi.nlm.nih.gov
. Ta primer ponazarja, kako lahko TDA v biomedicinskih podatkih odkrije pomembne skrite vzorce, ki jih tradicionalne metode ne zaznajo, bodisi zaradi šuma bodisi zaradi kompleksnosti podatkov
pmc.ncbi.nlm.nih.gov
. Topološka analiza podatkov se tako vse bolj izkazuje kot obetaven pristop za robustno analizo kompleksnih, visokodimenzionalnih podatkovnih množic v znanosti in inženirstvu.

\begin{algorithm}
\caption{Mapper}\label{alg:mapper2}
\begin{algorithmic}
\State{} \textbf{Vhod:} Zbirka (množica) podatkov $X$ z metriko, funkcija $f: X \rightarrow \mathbb{R}$ (ali $\mathbb{R}^d$), in pokritje $\mathcal{U}$ prostora $f(X)$.
\State{} \textbf{Algoritem:}
\For{each $U \in \mathcal{U}$}
    \State{} Razdeli $f^{-1}(U)$  v skupine $C_{U,1}, \ldots, C_{U,k_U}$.
    \State{} Izračunaj živec pokritja $X$, ki ga definira $\{C_{U,1}, \ldots, C_{U,k_U}\}$ for each $U \in \mathcal{U}$.
\EndFor{}
\State{} \textbf{Izhod:} Simplicialni kompleks, živec (pogosto graf za dobro izbrana pokritja):
\State{} \quad – Vozlišče $v_{U,i}$ za vsako skupino $C_{U,i}$.
\State{} \quad – Povezava $v_{U,i}$ in $v_{U',j}$ če $C_{U,i} \cap C_{U',j} \neq \emptyset$.
\end{algorithmic}
\end{algorithm}

\begin{algorithm}
\caption{Algoritem Mapper}\label{alg:mapper}
\begin{algorithmic}[1]
\State \textbf{Vhod:} Množica podatkov $X$ z metrikom, 
    zvezna funkcija $f: X \rightarrow \mathbb{R}$ (ali $f: X \rightarrow \mathbb{R}^d$), 
    in odprto pokritje $\mathcal{U} = \{U_\alpha\}$ prostora $f(X)$.
\State \textbf{Postopek:}
\For{vsak $U_\alpha \in \mathcal{U}$}
    \State Izračunaj preddsliko $f^{-1}(U_\alpha) \subseteq X$.
    \State Na $f^{-1}(U_\alpha)$ uporabi metodo gručenja (npr. single-linkage) in razdeli v gruče $C_{\alpha,1}, \ldots, C_{\alpha,k_\alpha}$.
\EndFor
\State \textbf{Zgradi simplicialni kompleks (živec pokritja):}
\State \quad Vozlišča ustrezajo posameznim gručam $C_{\alpha,i}$.
\State \quad Poveži dve vozlišči $C_{\alpha,i}$ in $C_{\beta,j}$ z robom, če velja $C_{\alpha,i} \cap C_{\beta,j} \neq \emptyset$.
\State \textbf{Izhod:} Simplicialni kompleks (pogosto graf), ki povzame topološko strukturo podatkov.
\end{algorithmic}
\end{algorithm}

\chapter{Mapper algoritem}
Algoritem Mapper je orodje za ekstrakcijo globalnih značilnosti iz visokodimenzionalnih podatkov, ki preoblikuje zapletene podatkovne množice v preproste simplicialne komplekse. Ti simplicialni kompleksi omogočajo kompakten globalni prikaz podatkov, neodvisno od individualnih razdalj, kotov ali točk. 
Algoritem ustvarja graf, ki odraža topološko strukturo izvornega oblaka točk. Začne z izbiro zvezne funkcije, ki se imenuje filter, $f : X \rightarrow Z$, ki na primer projicira prostor $X$ na podprostor $Z$. Nato določi končno odprto pokritje $\mathcal{U} = \{U_i\}_{i \in I}$ prostora $Z$. S pomočjo algoritma za združevanje v gruče se neodvisno obdela vsaka praslika $f^{-1}(U_i)$, od koder dobimo odprto pokritje $\mathcal{V} = \{V_j\}_{j \in J}$ oblaka točk. Končni Mapperjev graf ali simplicialni kompleks je opredeljen z živcem $\mathcal{V}$, nizom podmnožic $\mathcal{K}$ iz množice $J$ z nepraznimi presečišči. Vsaka podmnožica z $n+1$ indeksi predstavlja $n$-simpleks, kjer enočlenske podmnožice tvorijo vozlišča grafa, dvodelne pa njegove robove.
\begin{comment}
    https://danedmiston.github.io/home_page/assets/Mapper.pdf
\end{comment}

\cite{Langenbahn2022}
\section{Algoritem}
V nadaljevanju so opisani koraki algoritma Mapper:
\begin{algorithm}
\caption{Mapper algoritem}\label{alg:cap}
\begin{algorithmic}
\State \textbf{Vhod:} Zbirka (množica) podatkov $X$ z metriko, funkcija $f: X \rightarrow \mathbb{R}$ (ali $\mathbb{R}^d$), in pokritje $\mathcal{U}$ prostora $f(X)$.
\State \textbf{Algoritem:}
\For{each $U \in \mathcal{U}$}
    \State Razdeli $f^{-1}(U)$  v skupine $C_{U,1}, \ldots, C_{U,k_U}$.
    \State Izračunaj živec pokritja $X$, ki ga definira $\{C_{U,1}, \ldots, C_{U,k_U}\}$ for each $U \in \mathcal{U}$.
\EndFor
\State \textbf{Izhod:} Simplicialni kompleks, živec (pogosto graf za dobro izbrana pokritja):
\State \quad - Vozlišče $v_{U,i}$ za vsako skupino $C_{U,i}$.
\State \quad - Povezava $v_{U,i}$ in $v_{U',j}$ če $C_{U,i} \cap C_{U',j} \neq \emptyset$.
\end{algorithmic}
\end{algorithm}

\chapter{Zaključek in rezultati}

V tem poglavju si bomo pogledali metodo vizualizacije nevronske mreže in kakšne rezultate dobimo. Pri tem bomo sledili~\cite{Gabella2021}.

Pri gradnji Mapper grafa iz nevronske mreže lahko za vsak sloj \(i\) z \(N_i\) nevroni, spremljamo evolucijo uteži nevronov, ki vstopajo vanj. Z drugimi besedami spremljamo stolpce v matriki uteži med nivojema \(i-1\) in \(i\). Nato pa za vsak korak v procesu učenja, ki ga izvedemo z metodo gradientnega spusta dobimo \(N_i\) vektorjev uteži velikosti \(N_{i-1}\), kar interpretiramo kot \(N_i\) točk v prostoru dimenzije \(N_{i-1}\). Ob koncu učenja dobimo oblak točk z \(\text{št.\ korakov} \times N_i\) točk, ki so dimenzije \(N_{i-1}\). Točke so nato s PCA projekcijo projicirane v 2-dimenzionalni prostor. Od tod potem z Mapper algoritmom zgradimo graf.

Rezultati, ki jih bomo predstavili so povzeti iz članka~\cite{Gabella2021}. Pogledali si bomo Mapper graf nevronske mreže, ki je bila zgrajena na podatkovni zbirki MNIST.\ Gre za veliko podatkovno zbirko ročno napisanih števk (slik velikosti 28$\times$28), ki se pogosto uporablja za učenje in testiranje metod strojnega učenja.
Nevronska mreža ima dva skrita nivoja s 100 nevroni in sigmoidno aktivacijsko funkcijo. Začetne vrednosti uteži so bile nastavljene na naključne majhne vrednosti. Za filter funkcijo je bila uporabljena $L^2$ norma in DBSCAN kot algoritem gručenja.

\begin{figure}[H]
  \centering
  \includegraphics[width=1\linewidth]{resources/mapper-minst.png}
  \caption{Evolucija uteži učenja nevronske mreže. Vir:~\cite{Gabella2021}}\label{fig:evolution}
\end{figure}

Z modro barvo so predstavljene uteži na začetku učenja in z rumeno na koncu. Opazimo, da se graf v zadnjem sloju razveji na 10 vej, kjer vsaka predstavlja eno izmed desetih števk. V skritem nivoju pa lahko opazimo, da se uteži nekaj časa razvijajo v enaki smeri in se šele nato razvejijo, kar bi lahko bila posledica PCA projekcije. Omenimo še, da so smeri razvejanja v različnih poskusih učenja ostale skoraj konstantne. V zadnjem sloju opazimo, da se graf že na začetku usmeri na dve nasprotni veji, ki se nato še dalje razvejita. Vidno je tudi, da so si nekatere števke zelo podobne, recimo osmica in trojka sta si zelo sorodni.


\cleardoublepage

%%%%%%%%%%%%%%%%%%%%%%%%%%%%%%%%%%%%%%%%
% literatura
%%%%%%%%%%%%%%%%%%%%%%%%%%%%%%%%%%%%%%%%
%\addcontentsline{toc}{chapter}{Literatura}


%\printbibliography[heading=bibintoc,type=article,title={Članki v revijah}]
%https://www.overleaf.com/project/609ce2055f917cb2f776732e
%\printbibliography[heading=bibintoc,type=inproceedings,title={Članki v zbornikih}]

%\printbibliography[heading=bibintoc,type=incollection,title={Poglavja v knjigah}]

\printbibliography[heading=bibintoc]%,title={Celotna literatura}]

\end{document}

